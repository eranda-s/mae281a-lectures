\mainmatter%
\setcounter{page}{1}

\lectureseries[\course]{\course}

\auth[\lecAuth]{Lecturer: \lecAuth\\ Scribe: \scribe}
\date{February 9, 2010}

\setaddress%

% the following hack starts the lecture numbering at 1
\setcounter{lecture}{10}
\setcounter{chapter}{10}

\lecture{Invariance Theory and Linearization}

\section{Invariance Theory}
Invariance concerns a function, $\dot{x}=f(x)$, converging to a set and then staying inside that set.

\begin{example}
Using the dynamics of a pendulum previously described in Example~\ref{ex:03pendulum} consider the Lyapunov function with derivative
$$\dot{V} = \frac{d}{dt}\left\{\frac{g}{l}\left(1-\cos x_1\right) + \frac{1}{2}x_2^2\right\} = -\frac{k}{m}x_2^2$$
which is negative semidefinite.
This does not result in asymptotic stability because of the negative semidefinite character of the derivative.
However, we do get $\dot{V}=0$ on $x_2=0$.

Suppose that $x_2(t)\triangleq0$.
This would lead to
$$\dot{x}_2 = -\frac{g}{l}\sin x_1 - \frac{k}{m}x_2 \Rightarrow \sin x_1(t) \triangleq 0.$$
Thus $\dot{V}$ can be maintained only at the origin and $x(t)\to0$.
$\lozenge$
\end{example}

\begin{theorem}{Barbashin-Krasovskii}
Suppose $V$ is positive definite and $\dot{V}\leq0 \forall x\in\mathcal{D}$.
Let $\mathcal{S}=\{x\in\mathcal{D} | \dot{V} = 0\}$ and suppose that no solution can stay forever in $\mathcal{S}$ other than $x(t)\triangleq0$.
Then it follows that $x=0$ is asymptotically stable.
\end{theorem}

\begin{theorem}{LaSalle's Invariance Principle}
Let $\Omega$ be a compact positively invariant set of $\dot{x}=f(x)$.
Let $V:\Omega\to\mathbb{R}$ be $\mathcal{C}^1$ such that $\dot{V}\leq0 \forall x \in \Omega$.
Let the set $\mathcal{E}=\{x\in\Omega | \dot{V}=0\}$ and let $\mathcal{M}$ be the largest invariant set contained in $\mathcal{E}$.
Then $x(0)\in\Omega \Rightarrow x(t)\to\mathcal{M}$ as $t\to\infty$.
\end{theorem}
Note that all Lyapunov level sets are positive invariant sets.
Also, there are no restriction on $V>0$.
However, it is \textit{very} rare to see that in practice.
Additionally, note that not all points in $\mathcal{M}$ are attractors but \textit{some} points are attractors.

\begin{example}
Let the system dynamics be given by
\begin{align*}
\dot{x} &= -|x|x + (1-|x|)xy \\
\dot{y} &= -\tfrac{1}{8}(1-|x|)x^2.
\end{align*}
Consider the Lyapunov function and derivative
\begin{align*}
V &= x^2 + 8y^2 \\
\dot{V} &= -2x^2.
\end{align*}
This gives $\mathcal{E}=\{x=0\}$ as the equilibrium set.
Since $x=0$ is an equilibrium set we have $\mathcal{M}=\mathcal{E}$.
By LaSalle's theorem
\begin{align*}
\left[\begin{array}{c} x(t) \\ y(t) \end{array}\right] \to\mathcal{M} = \{x=0\}.
\end{align*}
$\lozenge$
\end{example}

\begin{example}
Let the system dynamics be given by
\begin{align*}
\dot{x} &= -x+xz+y(1-y) \\
\dot{y} &= -x(1-y) \\
\dot{z} &= -x^2.
\end{align*}
Consider the Lyapunov function and derivative
\begin{align*}
V &= \tfrac{1}{2}(x^2+y^2+z^2) \\
\dot{V} &= x\dot{x} + y\dot{y} + z\dot{z} = -x^2.
\end{align*}
Again we have that the equilibrium set is $\mathcal{E}=\{x=0\}$.
The equilibria are located at $x=0$ and $y=0$ or $y=1$.
This gives
$$\mathcal{M} = \{x=0,y=0\} \cup \{x=0,y=1\}.$$
$\lozenge$
\end{example}

\section{Linear Systems and Linearization}
Recall from linear system theory that for the system $\dot{x}=Ax$ a solution for $x$ takes the form $x(t)=e^{At}x(0)$.
Simple substitution shows that $x(t)$ is indeed a solution.

\begin{theorem}
Let $x=0$.
\begin{enumerate}
\item System is stable if and only if $\text{Re}(\lambda_i)\leq0$ and $(\text{Re}(\lambda_i)=0\Rightarrow m_i=1)\forall i$.
\item System is asymptotically stable if and only if $\text{Re}(\lambda_i)<0 \forall i$.
In this case the system matrix $A$ is called a Hurwitz matrix.
\end{enumerate}
\end{theorem}

For linear systems reasonable Lyapunov functions are quadratic where $V(x)=x^T Px$, $P=P^T>0$.
This leads to
\begin{align}
\label{eq:10linearLyapDeriv}
\dot{V} = x^T(PA+A^T P)x = -x^T Qx
\end{align}
This is what we want to find to show that $A$ is stable for all $x$.

\begin{theorem}
The system matrix $A$ is Hurwitz if and only if $\forall Q=Q^T>0\exists$ a unique $P=P^T>0$ such that $PA+A^T P = -Q$.
\end{theorem}

\begin{proof}
Let
$$P=\int_0^\infty e^{A^T t}Qe^{At}dt.$$
The other direction of the if and only if follows trivially from the Lyapunov function in (\ref{eq:10linearLyapDeriv}).
\end{proof}
