\mainmatter%
\setcounter{page}{1}

\lectureseries[\course]{\course}

\renewcommand{\scribe}{Thomas Denewiler, Nick Morozovsky}
\auth[\lecAuth]{Lecturer: \lecAuth\\ Scribe: \scribe}
\renewcommand{\scribe}{Thomas Denewiler}
\date{January 21, 2010}

\setaddress%

% the following hack starts the lecture numbering at 1
\setcounter{lecture}{5}
\setcounter{chapter}{5}

\lecture{Course Introduction}

\section{Sensitivity Equations}
This corresponds to Chapter 3.3 of Khalil.
We now want to look at systems where

\begin{equation*}
\dot{x} = f(t,x,\lambda)
\end{equation*}

By sensitivity we mean the derivative.
We are interested in how uncertainties or changes in a system's properties (such as mass, length, etc.) will affect the system behavior.

We want

\begin{equation*}
x_\lambda = \frac{\partial x}{\partial\lambda}
\end{equation*}

Consider the nominal case where $\lambda=\lambda_0$ and the nominal system $\dot{x}=f(t,x,\lambda_0)$.
Assuming that there exists a unique solution over the range $[t_0,t_1]$ this leads to
\begin{align*}
x(t,\lambda) &= x_0 + \int_{t_0}^t f(x,x(s,\lambda),\lambda)ds \\
x_\lambda(t,\lambda) &= \int_{t_0}^t \left[ \frac{\partial f}{\partial x}(s,x(s,\lambda),\lambda)x_\lambda(s,\lambda) + \frac{\partial f}{\partial\lambda}(s,x(s,\lambda),\lambda) \right]ds
\end{align*}
where
\begin{align*}
x_\lambda(t,\lambda) = \frac{\partial x(t,\lambda)}{\partial\lambda}, \qquad \frac{\partial x_0}{\partial\lambda} = 0
\end{align*}
Differentiating with respect to time it can be seen that $x_\lambda(t,\lambda)$ satisfies

\begin{equation*}
\frac{\partial}{\partial t}x_\lambda(t,\lambda) = A(t,\lambda)x_\lambda(t,\lambda) + B(t,\lambda), \qquad x_\lambda(t_0,\lambda) = 0
\end{equation*}

where

\begin{equation*}
A(t,\lambda) = \left.\frac{\partial f(t,x,\lambda)}{\partial x}\right|_{x=x(t,\lambda)}, \qquad B(t,\lambda) = \left.\frac{\partial f(t,x,\lambda)}{\partial\lambda}\right|_{x=x(t,\lambda)}
\end{equation*}


The \textit{sensitivity equation} is denoted by $S(t) = x_\lambda(t,\lambda)$ and is the unique solution of the equation

\begin{equation*}
\dot{S}(t) = A(t,\lambda_0)S(t) + B(t,\lambda_0), \qquad S(t_0) = 0
\end{equation*}

We need a solution of $\dot{x}=f(t,x,\lambda_0)$ to compute $S(t)$ with $S(t_0)=0$.
To get there we take a Taylor's series expansion about $\lambda_0$ and find

\begin{equation*}
x(t,\lambda) \approx x(t,\lambda_0) + S(t)(\lambda-\lambda_0)
\end{equation*}

This is only a valid approximation over a small range of $\lambda-\lambda_0$.

\section{Comparision Principle}
This corresponds to Chapter 3.4 of Khalil.
We are looking at differential inequalities such that

\begin{equation*}
\dot{v}\leq f(t,v)
\end{equation*}

where $v$ is a scalar and $f$ is scalar-valued (not a vector field).
Now consider the differential equation

\begin{equation*}
\dot{u} = f(t,u)
\end{equation*}

using the same function $f$ and recall Gr\"onwalls Lemma in \S\ref{sec:04gronwall} which showed that

\begin{equation*}
y(t) \leq \lambda(t) + \int_a^t\mu(s)y(s)ds
\end{equation*}

Then we have the following lemma.

\begin{lemma}
If $v(t_0)\leq u(t_0)$ then $v(t)\leq u(t)$.
\end{lemma}

\begin{example}
Let
\begin{align*}
\dot{x} &= -x(1+x^2+y^2) \\
\dot{y} &= -y(1-x^2)+\sin(t) \\
r &= \sqrt{x^2+y^2} \\
v &= r^2
\end{align*}
This leads to
\begin{align*}
\dot{v} &= \frac{d}{dt}(x^2+y^2) \\
&= 2(x\dot{x}+2y\dot{y}) \\
&= 2(-x^2-x^4-x^2y^2-y^2+x^2y^2+y\sin(t)) \\
&= 2(-v-r^4\cos^4\theta+r\sin\theta\sin t)
\end{align*}
Then we get
\begin{align*}
\dot{r} &= -r-r^3\cos^4\theta+\cos\theta\sin t \\
\Rightarrow \dot{r} &\leq -r+1 \\
\Rightarrow r(t) &\leq e^{-t}r_0 + \int_0^t e^{-(t-\tau)}1d\tau \\
\Rightarrow r(t) &\leq e^{-t}r_0 + 1 - \underbrace{e^{-t}}_{\to0} \\
&\leq e^{-t}r_0 + 1 \\
\Rightarrow \sqrt{x^2(t)+y^2(t)} &\leq e^{-t}\sqrt{x_0^2+y_0^2} + 1
\end{align*}
This shows that the energy of the system decays to within the unit circle.
$\lozenge$
\end{example}
